\documentclass[conference]{IEEEtran}
\usepackage[utf8]{inputenc}
\usepackage[T1]{fontenc}

\begin{document}

\title{Implémentation du jeu Hex via la Programmation Orientée Objet et stratégies de décision en C++}

\author{
    \IEEEauthorblockN{1\textsuperscript{st} Jair Anderson Vasquez Torres}
    \IEEEauthorblockA{\textit{Ingénieur Degree Programme STIC} \\
    \textit{ENSTA Paris}\\
    Paris, France \\
    jair-anderson.vasquez@ensta.fr}
    \and
    \IEEEauthorblockN{2\textsuperscript{nd} Santiago Florido Gomez}
    \IEEEauthorblockA{\textit{Ingénieur Degree Programme STIC} \\
    \textit{ENSTA Paris}\\
    Paris, France \\
    santiago.florido@ensta-paris.fr}
}

\maketitle

\begin{abstract}
Ce projet implémente en C++ un système complet pour le jeu Hex, conçu principalement comme un exercice de Programmation Orientée Objet. La solution organise le domaine du jeu au moyen de classes qui encapsulent le plateau, l'état, les coordonnées et les règles, permettant la génération de coups légaux et la vérification de victoire au moyen d'un parcours BFS sur les connexions des pions. Sur cette base s'intègrent des stratégies de décision (p. ex., Negamax avec hachage et table de transposition) et une évaluation de positions découplée du moteur, qui peut être heuristique ou s'appuyer sur un modèle neuronal de valeur intégré à l'exécutable (exportable en TorchScript), sans lier la conception à une architecture spécifique. De plus, une interface graphique (GUI) en SFML a été développée pour faciliter l'interaction, la visualisation du plateau et les tests du comportement des agents.
\end{abstract}

\begin{IEEEkeywords}
Hex game, C++, Object-Oriented Programming, Negamax, Transposition Table, TorchScript, SFML.
\end{IEEEkeywords}


\section{Détecteurs}

\subsection{Q4}

La fonction d'intérêt de Harris se base sur le fait d'assigner une valeur à chaque pixel. Cette valeur mesure
à quel point ce pixel est une \emph{intersection}, c'est-à-dire un \emph{coin}.

Dire qu'un pixel est un coin signifie qu'on cherche à quel point il ressemble à l'intersection de deux bords.
Autrement dit, à quel point ce pixel correspond à un endroit où l'image change fortement dans deux directions
perpendiculaires. C'est utile, parce qu'avec une petite fenêtre de pixels (la fenêtre $W$), on peut estimer
à quel point ce qu'on observe change au niveau de la structure.

Intuitivement, si on déplace légèrement la fenêtre $W$ autour d'un pixel, on observe :
\begin{itemize}
    \item \textbf{Zone plane (sans texture)} : la fenêtre change très peu $\rightarrow$ ce n'est pas un coin.
    \item \textbf{Bord} : la fenêtre change peu en se déplaçant \emph{le long} du bord, mais change beaucoup 
    en le \emph{traversant}.
    \item \textbf{Coin} : la fenêtre change beaucoup dans presque n'importe quelle direction.
\end{itemize}

Ce que fait Harris, c'est utiliser les gradients $I_x$ et $I_y$ pour savoir à quel point l'intensité change
 selon $x$ et selon $y$.
Si autour du pixel il y a un changement fort dans une seule direction, on est plutôt sur un bord ; mais si 
les changements forts sont dans deux
directions, on parle d'un coin. Ainsi, $Theta$, la réponse de Harris, est définie pixel par pixel : plus 
$Theta$ est grande, plus le pixel est un \emph{coin}.

\end{document}
