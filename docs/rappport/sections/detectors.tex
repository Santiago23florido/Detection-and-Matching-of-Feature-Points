\section{Détecteurs}

La fonction d'intérêt de Harris se base sur le fait d'assigner une valeur à chaque pixel. Cette valeur mesure
à quel point ce pixel est une \emph{intersection}, c'est-à-dire un \emph{coin}.

Dire qu'un pixel est un coin signifie qu'on cherche à quel point il ressemble à l'intersection de deux bords.
Autrement dit, à quel point ce pixel correspond à un endroit où l'image change fortement dans deux directions
perpendiculaires. C'est utile, parce qu'avec une petite fenêtre de pixels (la fenêtre $W$), on peut estimer
à quel point ce qu'on observe change au niveau de la structure.

Intuitivement, si on déplace légèrement la fenêtre $W$ autour d'un pixel, on observe :
\begin{itemize}
    \item \textbf{Zone plane (sans texture)} : la fenêtre change très peu $\rightarrow$ ce n'est pas un coin.
    \item \textbf{Bord} : la fenêtre change peu en se déplaçant \emph{le long} du bord, mais change beaucoup 
    en le \emph{traversant}.
    \item \textbf{Coin} : la fenêtre change beaucoup dans presque n'importe quelle direction.
\end{itemize}

Ce que fait Harris, c'est utiliser les gradients $I_x$ et $I_y$ pour savoir à quel point l'intensité change
 selon $x$ et selon $y$.
Si autour du pixel il y a un changement fort dans une seule direction, on est plutôt sur un bord ; mais si 
les changements forts sont dans deux
directions, on parle d'un coin. Ainsi, $Theta$, la réponse de Harris, est définie pixel par pixel : plus 
$Theta$ est grande, plus le pixel est un \emph{coin}.